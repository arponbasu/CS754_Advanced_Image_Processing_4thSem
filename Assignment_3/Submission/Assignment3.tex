\documentclass[a4paper,11pt]{article}
\usepackage{filecontents}

\usepackage[utf8]{inputenc}
\usepackage[english]{babel}
\usepackage{graphicx, array, blindtext}
\usepackage[colorinlistoftodos]{todonotes}
\DeclareUnicodeCharacter{2212}{-}
\usepackage [a4 paper , hmargin = 1.2 in , bottom = 1.5 in] {geometry}
\usepackage [parfill] {parskip}

\usepackage{enumitem}
\usepackage{amsmath}
\usepackage{amsthm}

\usepackage{nameref}
\usepackage{amssymb}
\usepackage [linesnumbered, ruled, vlined] {algorithm2e}
\usepackage{listings}
\usepackage{xcolor}
\usepackage{floatrow}
\usepackage{siunitx}
\usepackage{cancel}
\usepackage{fancyhdr}
\usepackage{graphicx}
\usepackage{verbatim}
\usepackage[document]{ragged2e}

\renewcommand{\footrulewidth}{0.4pt}
\newtheorem{definition}{Definition}
\numberwithin{definition}{section}
\newtheorem{mytheorem}{Theorem}
\numberwithin{mytheorem}{subsection}
\newcommand{\notimplies}{\;\not\!\!\!\longrightarrow}  
\newcommand\norm[1]{\left\lVert#1\right\rVert}
\pagestyle{fancy}
\fancyhf{}
\rhead{CS754 Assignment 3}
\lhead{200050013-200050130}
\fancyfoot[C]{Page \thepage}
\usepackage{subcaption}
\usepackage{listings}


\usepackage{hyperref}
\urlstyle{same}
\hypersetup{pdftitle={main.pdf},
    colorlinks=false,
    linkbordercolor=red
}
\usepackage{array}
\usepackage{listings,chngcntr}

\begin{document}
\centering{

\title{\fontsize{150}{60}{CS754 Assignment 3 Report}}

\author{
Arpon Basu \\ Shashwat Garg }
}

\date{Spring 2022}
\maketitle

\justifying
\tableofcontents

\newpage
\justifying
\section*{Introduction}

Welcome  to our report on CS754 Assignment 3. We have tried to make this report comprehensive and self-contained. We hope reading this would give you a proper flowing description of our work, methods used and the results obtained.

Hope you enjoy reading the report. Here we go!


\section{Problem 1}


\section{Problem 2}


\section{Problem 3}


\section{Problem 4}


\section{Problem 5}

We know that Radon Transform is given by-
$$R_\theta(f) =g(\rho, \theta)= \int_{-\infty}^{+\infty}f(\rho cos\theta - zsin\theta,\rho sin \theta + z cos\theta)dz $$
We can write the same as-
$$R_\theta(f) =g(\rho, \theta)= \int_{-\infty}^{+\infty}\int_{-\infty}^{+\infty}f(x,y)\delta(xcos\theta+ysin\theta -\rho)dxdy $$
Let the scaled image be denoted by $h(x.y) = f(ax, ay)$. This is the same image as original, but scaled by a factor of a, in both x and y directions.

We can write the same Radon Transform as-
$$ R_\theta(h) =g'(\rho, \theta)= \int_{-\infty}^{+\infty}\int_{-\infty}^{+\infty}h(x,y)\delta(xcos\theta+ysin\theta -\rho)dxdy $$
$$ R_\theta(h) =g'(\rho, \theta)= \int_{-\infty}^{+\infty}\int_{-\infty}^{+\infty}f(ax,ay)\delta(xcos\theta+ysin\theta -\rho)dxdy $$
$$ R_\theta(h) =g'(\rho, \theta)= \int_{-\infty}^{+\infty}\int_{-\infty}^{+\infty}f(x',y')\delta\bigg(\frac{x'cos\theta+y'sin\theta -a\rho}{a}\bigg)\frac{dx'}{a}\frac{dy'}{a} $$

Since $\delta(ax) = \delta(x)/a$, we get-
$$ R_\theta(h) =g'(\rho, \theta)= \frac{1}{a}\int_{-\infty}^{+\infty}\int_{-\infty}^{+\infty}f(x',y')\delta(x'cos\theta+y'sin\theta -a\rho)dx'dy' $$
$$ R_\theta(h) =g'(\rho, \theta)= \frac{1}{a}g(a\rho, \theta) $$

Thus, we can see that the Radon transform of the scaled image is also scaled by a factor of $a$ in the size of projection, but the intensity of each projection has reduced by $a$ as well.
 








\section{Problem 6}












\end{document}
