\documentclass[a4paper,11pt]{article}
\usepackage{filecontents}

\usepackage[utf8]{inputenc}
\usepackage[english]{babel}
\usepackage{graphicx, array, blindtext}
\usepackage[colorinlistoftodos]{todonotes}
\DeclareUnicodeCharacter{2212}{-}
\usepackage [a4 paper , hmargin = 1.2 in , bottom = 1.5 in] {geometry}
\usepackage [parfill] {parskip}

\usepackage{enumitem}
\usepackage{amsmath}
\usepackage{amsthm}

\usepackage{nameref}
\usepackage{amssymb}
\usepackage [linesnumbered, ruled, vlined] {algorithm2e}
\usepackage{listings}
\usepackage{xcolor}
\usepackage{floatrow}
\usepackage{siunitx}
\usepackage{cancel}
\usepackage{fancyhdr}
\usepackage{graphicx}
\usepackage{verbatim}
\usepackage[document]{ragged2e}

\renewcommand{\footrulewidth}{0.4pt}
\newtheorem{definition}{Definition}
\numberwithin{definition}{section}
\newtheorem{mytheorem}{Theorem}
\numberwithin{mytheorem}{subsection}
\newcommand{\notimplies}{\;\not\!\!\!\longrightarrow}  
\newcommand\norm[1]{\left\lVert#1\right\rVert}
\pagestyle{fancy}
\fancyhf{}
\rhead{CS754 Assignment 2}
\lhead{200050013-200050130}
\fancyfoot[C]{Page \thepage}
\usepackage{subcaption}
\usepackage{listings}


\usepackage{hyperref}
\urlstyle{same}
\hypersetup{pdftitle={main.pdf},
    colorlinks=false,
    linkbordercolor=red
}
\usepackage{array}
\usepackage{listings,chngcntr}

\begin{document}
\centering{

\title{\fontsize{150}{60}{CS754 Assignment 2 Report}}

\author{
Arpon Basu \\ Shashwat Garg }
}

\date{Spring 2022}
\maketitle

\justifying
\tableofcontents

\newpage
\justifying
\section*{Introduction}

Welcome  to our report on CS754 Assignment 2. We have tried to make this report comprehensive and self-contained. We hope reading this would give you a proper flowing description of our work, methods used and the results obtained. Feel free to keep our code script alongside to know the exact implementation of our tasks. 

Hope you enjoy reading the report. Here we go!


\section{Problem 1}



\subsection{Given $\delta_{2s}=1$}

In this case, our RIP theorem inequality becomes-
$$0 \leq \norm{\Phi h}_2^2 \leq 2 \norm{h}_2^2$$
where $h$ is a 2s-sparse vector.

We know that $\delta_{2s}$ is the smallest number such that the above holds for all s-sparse vectors for the matrix $\Phi$.

This means that, unless the $\delta_{2s}=1$ is present to take care of the upper bound on $\norm{\Phi h}_2^2$, there will exist an s-sparse vector such that $\norm{\Phi h}_2^2=0$, which implies that $\Phi h =\boldsymbol{0}$. Denoting the columns of $\Phi$ as $C_1, C_2 \cdots C_n$ and the entries of $h$ as $a_1, a_2, \cdots a_n$, we can get that-
$$ \sum_{i=1}^n C_i \cdot a_i = 0 $$

If we focus only on the non-zero $a_i$, where $h \neq 0$, we can get a linear combination of 2s columns of $\Phi$ which sums up to 0. Thus, in this case, 2s columns of $\Phi$ are linearly dependent.

Hence, when $\delta_{2s}=1$, 2s columns of $\Phi$ may be linearly dependent.

\subsection{Triangle Inequality}

We are trying to solve the following-

$$ \min_{\tilde{x} \in \mathbb{R}^n} \norm{\tilde{x}}_1 ~where ~\norm{y - \Phi \tilde{x}}_2 \leq \epsilon $$

The solution we receive is called $x^*$ and the actual, true, unknown solution is called $x$. Now, note that our reconstruction demands that  $\norm{y - \Phi x^*}_2 \leq \epsilon$

This implies that $x^*$ will also obey the same property.

Secondly, in the premise of the Theorem, we assume that the noise magnitude is less than $\epsilon$. This is one of the factors we take into account while choosing the $\epsilon$ value. Hence, by choice of $\epsilon$ in noise, we get that
$\norm{y - \Phi x}_2 \leq \epsilon$.

Finally, with these two tools, we can use the Triangle Inequality as follows-
$$\norm{\Phi(x^* - x)}_2 = \norm{(\Phi x^* - y) + (y - \Phi x)}_2 \leq \norm{\Phi x^* - y}_2 + \norm{y-\Phi x}_2$$
Using the two arguments above, we can say that-
$$\norm{\Phi x^* - y}_2 + \norm{y-\Phi x}_2 \leq \epsilon + \epsilon = 2\epsilon$$
Thus, finally, we get that 
$$\norm{\Phi(x^* - x)}_2 \leq \norm{\Phi x^* - y}_2 + \norm{y-\Phi x}_2 \leq 2\epsilon$$
Hence, proved.























\section{Problem 2}
(a) First of all, note that $\boldsymbol{\Phi x = \Phi_S\widetilde{x}}$, since all other entries of $\boldsymbol{x}$ are zero, and hence all the corresponding columns of $\boldsymbol{\Phi}$ get nullified. Now, we introduce the notion of \textbf{pseudo-inverse} as mentioned in the question: For any matrix $A$ (which can be rectangular too), there exists a unique \textbf{Moore-Penrose pseudo-inverse} $A^{\dagger}$ which mimics many of the properties that the actual inverse satisfies. In particular, if it so happens that the columns of $A$ are linearly independent, then $A^{\dagger}A = I$. Also note that since we are assuming $\boldsymbol{\Phi}$ satisfies RIC of order $S$, it automatically implies that any $S$ columns of $\boldsymbol{\Phi}$ are linearly independent. Thus hereon we can assume WLOG that $\boldsymbol{\Phi^{\dagger}_S\Phi_S = I}$.\\
Armed with this knowledge, it's very easy to determine the oracular solution: Note that while deriving any \textbf{deterministic} theoretical solution, we must ignore the noise $\boldsymbol{\eta}$ as we have no precise way of quantifying it apart from an upper bound on it's norm. Thus, $\boldsymbol{y_{\mathrm{theoretical}} = \Phi x = \Phi_s \widetilde{x}}$, and thus 
$$\boldsymbol{\widetilde{x} = \Phi_S^{\dagger}y_{\mathrm{theoretical}}}$$
Note that we don't actually know the value of $\boldsymbol{y_{\mathrm{theoretical}}}$. This is a shortcoming that is fixed below.
\\
(b) Clearly 
\begin{gather*}
\boldsymbol{y = \Phi x + \eta = \Phi_S\widetilde{x} + \eta} \\
\Rightarrow \boldsymbol{\Phi^{\dagger}_S y = \Phi_S^{\dagger}\Phi_S \widetilde{x} + \Phi_S^{\dagger}\eta = \widetilde{x} + \Phi_S^{\dagger}\eta}
\end{gather*}
Treating $\boldsymbol{\Phi^{\dagger}_S y}$ as the ``actual" $\boldsymbol{x}$, we get that 
\begin{gather*}
    \boldsymbol{x = \widetilde{x} + \Phi^{\dagger}_S\eta}\\
    \Rightarrow\boldsymbol{\lVert x-\widetilde{x}\rVert_2 = \lVert\Phi_S^{\dagger}\eta\rVert_2 }
\end{gather*}
Now, from the theory of \textbf{Singular Value Decomposition} we recall that the largest singular value of any matrix $A$ could be framed as the maximum of the following optimization problem
$$\boldsymbol{\sigma_{\mathrm{max}} = \lVert A\rVert_2 = \mathrm{max}_{\lVert x\rVert_2 = 1}\lVert Ax\rVert_2}$$
Then it's a direct consequence of the above maximization that $\boldsymbol{\lVert\Phi_S^{\dagger}\eta\rVert_2 \leq \lVert\Phi_S^{\dagger}\rVert_2\lVert\eta\rVert_2}$, as desired.
\\
(c) From the theory of \textbf{Singular Value Decomposition} we know that the singular values of any matrix $A$ are the \textbf{square roots of the eigenvalues of $AA^*$} (note that the eigenvalues of $AA^*$ are non-negative by the Spectral theorem since $AA^*$ is Hermitian). Thus, to calculate the singular values of $\boldsymbol{\Phi_S^{\dagger}}$, we first evaluate the matrix $\boldsymbol{\Phi_S^{\dagger}}\boldsymbol{\Phi_S^{\dagger^{*}}}$ 
$$\boldsymbol{\Phi_S^{\dagger}\Phi_S^{\dagger^{*}} = ((\Phi_S^*\Phi_S)^{-1}\Phi_S)(\Phi_S^*(\Phi_S^*\Phi_S)^{-1}) = (\Phi_S^*\Phi_S)^{-1}}$$
Thus, $\boldsymbol{\Phi_S^{\dagger}\Phi_S^{\dagger^{*}} = (\Phi_S^*\Phi_S)^{-1}}$. Now, we utilise a linear algebra property which says that if $\lambda$ is an eigenvalue of an invertible matrix $A$, then $\frac{1}{\lambda}$ will be an eigenvalue of the matrix $A^{-1}$. \\
Since $\boldsymbol{\Phi_S^*\Phi_S}$ is obviously invertible, the eigenvalues of $\boldsymbol{(\Phi_S^*\Phi_S)^{-1}}$ are the reciprocals of the eigenvalues of $\boldsymbol{\Phi_S^*\Phi_S}$.\\
Now, since $\boldsymbol{\Phi}$ is RIC with order $|S| = 2k$, from the slides (\texttt{CS\_Theory, Pg.101/109}) we have that $\boldsymbol{\delta_{2k}}$ = max$\boldsymbol{(\lambda_{\mathrm{max}}-1, 1-\lambda_{\mathrm{min}})}$ where $\boldsymbol{\lambda_{\mathrm{max}}}$ is the maximum eigenvalue of $\boldsymbol{\Phi_{\Gamma}\Phi_{\Gamma}^*}$ and $\boldsymbol{\lambda_{\mathrm{min}}}$ is the minimum eigenvalue of $\boldsymbol{\Phi_{\Gamma}\Phi_{\Gamma}^*}$ over all possible sets $\boldsymbol{\Gamma\subseteq[n],\;|\Gamma| \leq |S|}$ (note that $\boldsymbol{\lambda_{\mathrm{max}}}$ and $\boldsymbol{\lambda_{\mathrm{min}}}$ both don't necessarily come from some same set $\boldsymbol{\Gamma}$). Then, considering the fact that $\boldsymbol{\Phi_{\Gamma}\Phi_{\Gamma}^*}$ and $\boldsymbol{\Phi_{\Gamma}^*\Phi_{\Gamma}}$ have the same non-zero eigenvalues, we see that every eigenvalue of $\boldsymbol{\Phi_S^*\Phi_S}$ lies in $\boldsymbol{[1-\delta_{2k}, 1+\delta_{2k}]}$ (because note that $S$ is just an instance of $\boldsymbol{\Gamma}$ and  $\boldsymbol{\delta_{2k}}$ = max$\boldsymbol{(\lambda_{\mathrm{max}}-1, 1-\lambda_{\mathrm{min}})}$ implies that all eigenvalues, are first bounded by $\boldsymbol{\lambda_{\mathrm{max}}}$ and $\boldsymbol{\lambda_{\mathrm{min}}}$ which are in turn bounded by $\boldsymbol{1-\delta_{2k}}$ and $\boldsymbol{1+\delta_{2k}}$. Refer to the solution of Q3 for an elaboration of the same), and consequently \textbf{EVERY} singular value of $\boldsymbol{\Phi_S^{\dagger}}$, which as we showed above were the square roots of the reciprocals of the eigenvalues of $\boldsymbol{\Phi_S^*\Phi_S}$, must lie within $\boldsymbol{[\frac{1}{\sqrt{1+\delta_{2k}}}, \frac{1}{\sqrt{1-\delta_{2k}}}]}$, as desired.
\\
(d) Note that Thm. 3 said that $\boldsymbol{\lVert x^* - x\rVert_2\leq\frac{C_0}{\sqrt{S}}\lVert x-\widetilde{x}\rVert_2 + C_1\epsilon}$. But we already know that $\boldsymbol{\lVert x-\widetilde{x}\rVert_2\leq\frac{\epsilon}{\sqrt{1-\delta_{2k}}}\leq\frac{\epsilon}{\sqrt{2-\sqrt{2}}}}$, since $\boldsymbol{\delta_{2k} \leq \sqrt{2} - 1}$ for Thm. 3 to hold.\\
Thus, $\boldsymbol{\lVert x^* - x\rVert_2\leq(\frac{C_0}{\sqrt{(2-\sqrt{2})S}} + C_1)\epsilon}$, and since $C_0$ and $C_1$ are known bounded constants we obtain that both the solutions are some constant factors of $\boldsymbol{\epsilon}$, and consequently some constant factors of each other.






\section{Problem 3}
% We establish a certain lemma first before proving the asked for result.
% \subsection{The Submatrix Lemma}
% \textbf{Claim: } Let $A \in \mathbb{K}^{m\times n}$ be a matrix and let $A_s \in \mathbb{K}^{m\times s}$ be a submatrix of $A$. Then all the eigenvalues of $A_s^{*}A_s$ are also eigenvalues of $A^{*}A$.\\
% \textbf{Proof: } Notice that the sampling of $A_s$ from $A$ can be done via a \textbf{selection matrix} $B\in\mathbb{K}^{n\times s}$ such that $A_s = AB$. Notice that the columns of $B$ are just ``one-hot" columns corresponding to the columns of $A$ which were sampled to make $A_s$.\\
% Then, note that $A_s^*A_sx = \lambda x$ $\Rightarrow$ $(AB)^*(AB)x = \lambda x$ $\Rightarrow$




\section{Problem 4}




\section{Problem 5}
\textbf{SHASHWAT CLARIFY:} SIR ASKED US TO DERIVE AN APPROPRIATE CONSTRAINT on $\epsilon^{\prime}$. MY PROOF BELOW WORKS WITHOUT ANY SUCH CONSTRAINTS. KINDLY LOOK INTO IT IF YOU HAVE TIME. PLUS MY PROOF IS TOO SHORT BUT IDK HOW TO EXTEND IT.\\
Let
\begin{gather*}
    \boldsymbol{x^* := \mathrm{min}_{x\in\mathbb{K}^n}(\lVert y-\Phi x\rVert^2_2 + \lambda\lVert x\rVert_1)}\\
    \boldsymbol{\epsilon^{\prime} := \lVert y-\Phi x^*\rVert_2}\\
    \boldsymbol{A_t := \{x_0: x_0 \in \mathbb{K}^n;\;\lVert y-\Phi x_0\rVert_2\leq t\}}
\end{gather*}
Thus the sample space of the problem P1 for any given $\boldsymbol{\epsilon}$ is $\boldsymbol{A_{\epsilon}}$. Now, choose $\boldsymbol{\epsilon = \epsilon^{\prime}}$. Thus, $\boldsymbol{\lVert y-\Phi x\rVert_2 \leq \epsilon^{\prime}}$ $\forall$ $\boldsymbol{x\in A_{\epsilon^{\prime}}}$. But since the value of $\boldsymbol{\lVert y-\Phi x\rVert_2}$ over $\boldsymbol{A_{\epsilon^{\prime}}}$ is lesser than or equal to $\boldsymbol{\epsilon^{\prime}}$, we have that their L1 norms should be greater than or equal to $\boldsymbol{\lVert x^*\rVert_1}$ because otherwise if there existed $\boldsymbol{x^{\prime}\in A_{\epsilon^{\prime}}}$ such that $\boldsymbol{\lVert y-\Phi x^{\prime}\rVert_2 \leq \epsilon^{\prime}}$ and $\boldsymbol{\lVert x^{\prime}\rVert_1 < \lVert x^*\rVert_1}$, then $\boldsymbol{J(x^{\prime}) < J(x^*)}$, contradicting the minimality of $\boldsymbol{x^*}$.\\
Thus $\boldsymbol{\lVert x^{\prime}\rVert_1 \geq \lVert x^{*}\rVert_1}$ $\forall$ $\boldsymbol{x^{\prime}\in A_{\epsilon^{\prime}}}$ (note that $\boldsymbol{x^*\in A_{\epsilon^{\prime}}}$ too). Thus, the set of vectors with minimum L1-norms in $\boldsymbol{A_{\epsilon^{\prime}}}$ includes $\boldsymbol{x^*}$, and consequently $\boldsymbol{x^*}$ is a minimizer for problem P1 too (although note that without additional information we can't say if $\boldsymbol{x^*}$ is the unique minimizer of P1 or not).


\section{Problem 6}
















\end{document}

This claim follows almost immediately from a result mentioned in the slides, which is that $\boldsymbol{\delta_{2k}}$ = max$\boldsymbol{(\lambda_{\mathrm{max}}-1, 1-\lambda_{\mathrm{min}})}$, where $\boldsymbol{\lambda_{\mathrm{max}}}$ is the maximum eigenvalue of $\boldsymbol{\Phi_{\Gamma}\Phi_{\Gamma}^*}$ and $\boldsymbol{\lambda_{\mathrm{min}}}$ is the minimum eigenvalue of $\boldsymbol{\Phi_{\Gamma}\Phi_{\Gamma}^*}$ over all possible sets $\boldsymbol{\Gamma\subseteq[n],\;|\Gamma| \leq |S|}$ (note that $\boldsymbol{\lambda_{\mathrm{max}}}$ and $\boldsymbol{\lambda_{\mathrm{min}}}$ both don't necessarily come from some same set $\boldsymbol{\Gamma}$). Now, for any $\boldsymbol{\lambda}$ which is an eigenvalue of some $\boldsymbol{}$