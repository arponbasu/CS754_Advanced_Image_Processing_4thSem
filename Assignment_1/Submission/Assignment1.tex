\documentclass[a4paper,11pt]{article}
\usepackage{filecontents}

\usepackage[utf8]{inputenc}
\usepackage[english]{babel}
\usepackage{graphicx, array, blindtext}
\usepackage[colorinlistoftodos]{todonotes}
\DeclareUnicodeCharacter{2212}{-}
\usepackage [a4 paper , hmargin = 1.2 in , bottom = 1.5 in] {geometry}
\usepackage [parfill] {parskip}

\usepackage{enumitem}
\usepackage{amsmath}
\usepackage{amsthm}
\usepackage{nameref}
\usepackage{amssymb}
\usepackage [linesnumbered, ruled, vlined] {algorithm2e}
\usepackage{listings}
\usepackage{xcolor}
\usepackage{floatrow}
\usepackage{siunitx}


\usepackage{cancel}
\usepackage{fancyhdr}
\usepackage{graphicx}
\usepackage{verbatim}
\usepackage[document]{ragged2e}

\renewcommand{\footrulewidth}{0.4pt}
\newtheorem{definition}{Definition}
\numberwithin{definition}{section}
\newtheorem{mytheorem}{Theorem}
\numberwithin{mytheorem}{subsection}
\newcommand{\notimplies}{\;\not\!\!\!\longrightarrow}  
\newcommand\norm[1]{\left\lVert#1\right\rVert}
\pagestyle{fancy}
\fancyhf{}
\rhead{CS754 Assignment 1}
\lhead{200050013-200050130}
\fancyfoot[C]{Page \thepage}
\usepackage{subcaption}
\usepackage{listings}


\usepackage{hyperref}
\urlstyle{same}
\hypersetup{pdftitle={main.pdf},
    colorlinks=false,
    linkbordercolor=red
}
\usepackage{array}
\usepackage{listings,chngcntr}

\begin{document}
\centering{

\title{\fontsize{150}{60}{CS754 Assignment 1 Report}}

\author{
Arpon Basu \\ Shashwat Garg }
}

% Replace photos after seed
% Align equations
% Explain each other



\date{Spring 2022}
\maketitle

\justifying
\tableofcontents

\newpage
\justifying
\section*{Introduction}

Welcome  to our report on CS215 Assignment 3. We have tried to make this report comprehensive and self-contained. We hope reading this would give you a proper flowing description of our work, methods used and the results obtained. Feel free to keep our code scripts alongside to know the exact implementation of our tasks. %The pictures included in the graphs folder are a part of this report as well.

We have referred to some sites on the web for finding the MATLAB implementations (generic documentation pages) and the same has been added in the references section. %general statistical knowledge needed for various parts of the assignment. 

In many places, to better give context to the place from which the questions could have arisen, some theoretical discussions have been engaged in.

Hope you enjoy reading the report. Here we go!


\section{Problem 1}
We have been given the following problem-

Let $\boldsymbol{\theta^{\star}}$ : $\textrm{min} \|\boldsymbol{\theta}\|_1$ such that $\|\boldsymbol{y}-\boldsymbol{\Phi \Psi \theta}\|_2 \leq \varepsilon$, where $\boldsymbol{x} = \boldsymbol{\Psi \theta}$ and $\boldsymbol{y} = \boldsymbol{\Phi x} + \boldsymbol{\eta}$. $\varepsilon$ is an upper bound on the magnitude of the noise vector $\boldsymbol{\eta}$.

Also, Theorem 3 states-\\If $\boldsymbol{\Phi}$ obeys the restricted isometry property with isometry constant $\delta_{2s} < \sqrt{2}-1$, then we have $\|\boldsymbol{\theta} - \boldsymbol{\theta^{\star}}\|_2 \leq C_1 s^{-1/2}\|\boldsymbol{\theta}-\boldsymbol{\theta_s}\|_1 + C_2 \varepsilon$ where $C_1$ and $C_2$ are functions of only $\delta_{2s}$ and where $\forall i \in \mathcal{S}, \boldsymbol{\theta_s}_i = \theta_i; \forall i \notin \mathcal{S}, \boldsymbol{\theta_s}_i = 0$.

\subsection{Trend of Error Bound with $s$}
This is not a discrepancy. In reality, the error bound becomes worse as the value of $s$ increases. The point is, we are only focusing on the effect of $s^{-1/2}$ and $||\boldsymbol{\theta}-\boldsymbol{\theta_s}||_1$. We must also see the change in $C_1$ and $C_2$. These constants increase as the value of $\delta_{2s}$ changes. Thus, as the value of $s$ increases, we observe that the bound on $delta_{2s}$ also increases which leads to an increase in the value of $C_1$ and $C_2$. Thus, we cannot claim that the error bound improves as the sparsity measure, $s$ increases in value.


\section{Problem 2}
\section{Problem 3}
\section{Problem 4}
\section{Problem 5}
\section{Problem 6}












\end{document}